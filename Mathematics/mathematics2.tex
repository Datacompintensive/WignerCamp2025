\documentclass{beamer}
\usefonttheme{serif} % default family is serif
\usepackage{graphicx}
\usepackage{xcolor}
\usepackage{colortbl}
\usepackage{multirow,tabularx}
\usepackage{tikz}
\usepackage{hyperref}
\usepackage{physics}
\usepackage{siunitx}

%% For the name of the chemical compounds
%\usepackage{mhchem}

% Code listing
\usepackage{listings}
\usepackage{xcolor}

% Define Bash syntax highlighting
\lstdefinelanguage{python}{
  morekeywords={if, then, else, fi, for, while, do, done, exit, echo, function},
  morecomment=[l]{\#},
  morestring=[b]",
  sensitive=true
}

% Define listing style
\lstset{
  language=python,
  basicstyle=\ttfamily\small,
  keywordstyle=\color{blue}\bfseries,
  commentstyle=\color{gray},
  stringstyle=\color{red},
  numbers=left,
  numberstyle=\tiny\color{gray},
  stepnumber=1,
  numbersep=5pt,
  breaklines=true,
  breakatwhitespace=true,
  frame=single,
  backgroundcolor=\color{lightgray!20},
  tabsize=2
}

\usepackage{amsthm}
\usepackage{amsmath}
%\usepackage{multilined}

%\usefonttheme{professionalfonts} % using non standard fonts for beamer
%\usefonttheme{serif} % default family is serif
%\usepackage{fontspec}
%\setmainfont{Times New Roman}

% Redefine example environment with separate numbering
\newtheorem{examplex}{Example}
\renewenvironment{example}[1][]{%
  \setbeamercolor{block title example}{use=structure,fg=white,bg=structure.fg!75!black}%
  \setbeamercolor{block body example}{parent=normal text,use=block title,bg=block title.bg!10!bg}%
  \begin{examplex}[#1]%
}{%
  \end{examplex}%
}
% For legend of equations
\newenvironment{conditions*}
  {\par{\abovedisplayskip}\noindent
   \tabularx{\columnwidth}{>{$}l<{$} @{\ : } >{\raggedright\arraybackslash}X}}
  {\endtabularx\par\vspace{\belowdisplayskip}}

% Definition block formatting remains the same
\setbeamertemplate{theorems}[numbered]
\setbeamercolor{block title}{use=structure,fg=white,bg=structure.fg!75!black}
\setbeamercolor{block body}{parent=normal text,use=block title,bg=block title.bg!10!bg}

% Package for figure captions
\usepackage[font=small,skip=2pt]{caption}
% Decrease space/padding after figure captions
\setlength{\belowcaptionskip}{-10pt}

% Define blue color
\definecolor{purple}{rgb}{0.3, 0.0, 0.557}
\definecolor{blue}{rgb}{0.0234375, 0.11328125, 0.609375}

\usepackage{multicol}
\usepackage{wrapfig}

% Use blue theme for the presentation
\setbeamercolor{frametitle}{fg=blue}
\setbeamercolor{title}{fg=blue}
\setbeamercolor{structure}{fg=blue}

\newcommand{\circumeq}{\mathrel{\widehat{=}}}
\newcommand\scalemath[2]{\scalebox{#1}{\mbox{\ensuremath{\displaystyle #2}}}}

% Define affiliations
\newcommand{\affiliation}[2]{#1\textsuperscript{\textcolor{blue}{#2}}}

\title{\textcolor{blue}{Mathematics II}}
\subtitle{\textcolor{blue}{Wigner Summer Camp \\ Data and Compute Intensive Sciences Research Group}}
\author{\textcolor{blue}{Balázs, Paszkál, Vince, Levente, Antal \\ Éva, Hajni}}

\date{\textcolor{blue}{7-11 July 2025}}


% Redefine footline to show only the slide number on the bottom left in blue
\setbeamertemplate{footline}
{
  \leavevmode%
  \hbox{%
  \begin{beamercolorbox}[wd=.1\paperwidth,ht=2.25ex,dp=1ex,left]{page number in head/foot}%
    \hspace{1em} \textcolor{blue}{\insertframenumber}%
  \end{beamercolorbox}}%Wigner Summer Camp

  \vskip0pt%
}

\begin{document}
	\begin{frame}
		\titlepage
		\begin{columns}
			\centering
			\column{0.3\textwidth}
			%\includegraphics[width=0.7\textwidth]{../img/logo.png}
			\column{0.3\textwidth}
			\centering
			\includegraphics[width=0.8\textwidth]{../img/logo.png}
			\column{0.3\textwidth}
			%\includegraphics[width=0.7\textwidth]{../img/logo.png}
			\centering
		\end{columns}
	\end{frame}

	\begin{frame}{What is the Determinant?}
		\begin{itemize}
		  \item The determinant is a scalar value associated with a square matrix \( A \).
		  \item It plays a key role in linear algebra:
		  \begin{itemize}
		    \item \( A \) is invertible if and only if \( \det(A) \neq 0 \).
		    \item \( |\det(A)| \) equals the volume scaling factor of the linear transformation defined by \( A \).
		  \end{itemize}
		  \item Determinants are widely used in solving linear systems, computing eigenvalues, and understanding geometric transformations.
		\end{itemize}
	\end{frame}

	\begin{frame}{Determinant of a \(2 \times 2\) Matrix}
		\begin{itemize}
		  \item For a \(2 \times 2\) matrix
		  \begin{equation}
		    A =
		    \begin{pmatrix} a & b \\ c & d \end{pmatrix},
		  \end{equation}
		  the determinant is given by
		  \begin{equation}
		    \det(A) = ad - bc. \label{eq:det2x2}
		  \end{equation}
		  \item Example:
		  \begin{equation}
		    \det
		    \begin{pmatrix} 1 & 2 \\ 3 & 4 \end{pmatrix} =
		    (1)(4) - (2)(3) = -2. \label{eq:det2x2_example}
		  \end{equation}
		\end{itemize}
	\end{frame}

	\begin{frame}{Exercise: \(2 \times 2\) Determinant}
		\begin{example}[Exercise]
		Compute the determinant of the matrix:
			\begin{equation}
			  A =
			  \begin{pmatrix} 5 & 7 \\ 2 & 6 \end{pmatrix}.
			\end{equation}
		\end{example}
	\end{frame}
		
	\begin{frame}{Solution: \(2 \times 2\) Determinant}
		\begin{example}[Solution]
		Using Equation~\eqref{eq:det2x2}:
			\begin{equation}
			  \det(A) =
			  (5)(6) - (7)(2) =
			  30 - 14 =
			  16.
			\end{equation}
		\end{example}
	\end{frame}

	\begin{frame}{Determinant of a \(3 \times 3\) Matrix}
	We compute the determinant of a \(3 \times 3\) matrix by expanding along the first row:
		\begin{equation}
		A =
		\begin{pmatrix} 
		a & b & c \\ 
		d & e & f \\ 
		g & h & i 
		\end{pmatrix}.
		\end{equation}
	
	The formula is:
		\begin{equation}
		\det(A) =
		a 
		\begin{vmatrix} 
		e & f \\ h & i 
		\end{vmatrix}
		- b 
		\begin{vmatrix} 
		d & f \\ g & i 
		\end{vmatrix}
		+ c 
		\begin{vmatrix} 
		d & e \\ g & h 
		\end{vmatrix}. \label{eq:det3x3_expansion}
		\end{equation}
	
	Each \(2 \times 2\) minor is the determinant of the smaller matrix obtained by deleting the corresponding row and column.
	\end{frame}

	\begin{frame}{Why does this work?}
	The determinant of \(A\) is defined recursively by expanding into \(2 \times 2\) minors:
		\begin{equation}
		\begin{vmatrix} 
		m_{11} & m_{12} \\ m_{21} & m_{22} 
		\end{vmatrix} =
		m_{11}m_{22} - m_{12}m_{21}.
		\end{equation}
	
	So:
		\begin{align}
		\det(A) =
		a 
		\begin{vmatrix} 
		e & f \\ h & i 
		\end{vmatrix}
		&- b 
		\begin{vmatrix} 
		d & f \\ g & i 
		\end{vmatrix}
		+ c 
		\begin{vmatrix} 
		d & e \\ g & h 
		\end{vmatrix}. 
		\end{align}
	
	\vspace{0.2cm}
	This is called the \emph{Laplace expansion}.
	
	Signs come from the cofactor sign pattern:
		\begin{center}
			\begin{tabular}{|c|c|c|}
			\hline
			\cellcolor{blue!20} \(+\) & \cellcolor{red!20} \(-\) & \cellcolor{blue!20} \(+\) \\ \hline
			\cellcolor{red!20} \(-\) & \cellcolor{blue!20} \(+\) & \cellcolor{red!20} \(-\) \\ \hline
			\cellcolor{blue!20} \(+\) & \cellcolor{red!20} \(-\) & \cellcolor{blue!20} \(+\) \\ \hline
			\end{tabular}
		\end{center}
	The determinant of an \(n \times n\) matrix is computed similarly: expanding along any row or column into \((n-1) \times (n-1)\) minors.
	\end{frame}

	\begin{frame}{Numerical Example}
	Compute:
		\begin{equation}
		A =
		\begin{pmatrix} 
		1 & 2 & 3 \\ 
		4 & 5 & 6 \\ 
		7 & 8 & 9 
		\end{pmatrix}.
		\end{equation}
	
	Expanding along the first row:
		\begin{align}
		\det(A) &=
		1 
		\begin{vmatrix} 
		5 & 6 \\ 8 & 9 
		\end{vmatrix}
		- 2 
		\begin{vmatrix} 
		4 & 6 \\ 7 & 9 
		\end{vmatrix}
		+ 3 
		\begin{vmatrix} 
		4 & 5 \\ 7 & 8 
		\end{vmatrix} \\
		&= 1(5\cdot9 -6\cdot8) -2(4\cdot9 -6\cdot7) +3(4\cdot8 -5\cdot7) \\
		&= 1(-3) -2(-6) +3(-3) \\
		&= -3 +12 -9 = 0.
		\end{align}
	
	The determinant of this matrix is \(0\), so it is singular (non-invertible).
	\end{frame}

	\begin{frame}{Exercise: Compute a \(3 \times 3\) Determinant}
		\begin{block}{Exercise}
		Compute the determinant of:
			\begin{equation}
			A =
			\begin{pmatrix} 
			2 & 1 & 3 \\ 
			0 & -1 & 4 \\ 
			5 & 2 & 0 
			\end{pmatrix}.
			\end{equation}
		Use expansion along the first row.
		\end{block}
	\end{frame}


	\begin{frame}{Solution: Exercise}
		\begin{block}{Solution}
			\begin{align}
			\det(A) &=
			2 
			\begin{vmatrix} 
			-1 & 4 \\ 2 & 0 
			\end{vmatrix}
			- 1 
			\begin{vmatrix} 
			0 & 4 \\ 5 & 0 
			\end{vmatrix}
			+ 3 
			\begin{vmatrix} 
			0 & -1 \\ 5 & 2 
			\end{vmatrix} \\
			&= 2((-1)(0) -4(2)) -1(0 -4(5)) +3((0)(2) -(-1)(5)) \\
			&= 2(0-8) -1(0-20) +3(0+5) \\
			&= -16 +20 +15 = 19.
			\end{align}
		Therefore, \(\det(A) = 19.\)
		\end{block}
	\end{frame}

	\begin{frame}{The Eigenvalue Problem — Step by Step}
	We want to find scalars \(\lambda\) and nonzero vectors \(\vec{v}\) such that:
		\begin{equation}
		A \vec{v} = \lambda \vec{v}.
		\end{equation}
	
	\textbf{Step 1:} Rewrite as
		\begin{equation}
		(A - \lambda I)\vec{v} = 0.
		\end{equation}
	
	\textbf{Step 2:} For a nontrivial solution (\(\vec{v} \neq 0\)), the matrix \(A - \lambda I\) must be singular:
		\begin{equation}
		\det(A - \lambda I) = 0.
		\end{equation}
	
	This is called the \emph{characteristic equation}.
	\end{frame}

	\begin{frame}{Solving the Eigenvalue Problem}
	\textbf{Step 3:} Expand \(\det(A - \lambda I) = 0\) to get a polynomial in \(\lambda\).
	
	\textbf{Step 4:} The roots of this polynomial are the eigenvalues.
	
	\textbf{Step 5:} For each eigenvalue \(\lambda\), solve:
		\begin{equation}
		(A - \lambda I)\vec{v} = 0
		\end{equation}
	to find the corresponding eigenvector(s) by computing the nullspace.
	\end{frame}

	\begin{frame}{Example: Eigenvalues and Eigenvectors}
	Given:
		\begin{equation}
		A =
		\begin{pmatrix} 2 & 1 \\ 1 & 2 \end{pmatrix}.
		\end{equation}
	
	\textbf{Step 1:} Compute \(A - \lambda I\):
		\begin{equation}
		A - \lambda I =
		\begin{pmatrix} 2-\lambda & 1 \\ 1 & 2-\lambda \end{pmatrix}.
		\end{equation}
	
	\textbf{Step 2:} Characteristic equation:
		\begin{align}
		\det(A-\lambda I) &= (2-\lambda)(2-\lambda) -1 \\
		&= \lambda^2 -4\lambda+3 =0.
		\end{align}
	
	\textbf{Step 3:} Roots:
		\begin{equation}
		\lambda_1 = 1, \quad \lambda_2 = 3.
		\end{equation}
	\end{frame}

	\begin{frame}{Example: Eigenvectors}
	For \(\lambda_2=3\):
		\begin{equation}
		A-3I =
		\begin{pmatrix} -1 & 1 \\ 1 & -1 \end{pmatrix}.
		\end{equation}
	We solve:
		\begin{equation}
		(A-3I)\vec{v} =
		\begin{pmatrix} -1 & 1 \\ 1 & -1 \end{pmatrix}
		\vec{v} =
		0.
		\end{equation}
	
	This gives \(v_1=v_2\), so an eigenvector is:
		\begin{equation}
		\vec{v}_1 =
		\begin{pmatrix}1\\1\end{pmatrix}.
		\end{equation}
	
	For \(\lambda_1=1\):
		\begin{equation}
		A-I =
		\begin{pmatrix}1 &1 \\1 &1\end{pmatrix}.
		\end{equation}
	
	This gives \(v_1=-v_2\), so an eigenvector is:
		\begin{equation}
		\vec{v}_2 =
		\begin{pmatrix}1\\-1\end{pmatrix}.
		\end{equation}
	\end{frame}

	\begin{frame}{Exercise: Find Eigenvalues and Eigenvectors}
		\begin{block}{Exercise}
		Find the eigenvalues and one eigenvector for each eigenvalue of the matrix:
			\begin{equation}
			A =
			\begin{pmatrix} 4 & 2 \\ 1 & 3 \end{pmatrix}.
			\end{equation}
		\end{block}
	\end{frame}

	\begin{frame}{Solution: Exercise}
	\textbf{Step 1:} Compute characteristic polynomial:
		\begin{align}
		\det(A-\lambda I) &=
		\begin{vmatrix}4-\lambda &2 \\1 &3-\lambda\end{vmatrix} =
		(4-\lambda)(3-\lambda)-2 \\
		&= \lambda^2 -7\lambda+10=0.
		\end{align}
	
	\textbf{Step 2:} Roots:
		\begin{equation}
		\lambda_1=5, \quad \lambda_2=2.
		\end{equation}
	
	\textbf{Step 3:} For \(\lambda_1=5\):
		\begin{equation}
		A-5I =
		\begin{pmatrix}-1&2\\1&-2\end{pmatrix}.
		\end{equation}
	Nullspace: \(v_1=2v_2\), so:
		\begin{equation}
		\vec{v}_1 =
		\begin{pmatrix}2\\1\end{pmatrix}.
		\end{equation}
	\end{frame}

	\begin{frame}{Solution 2}
	For \(\lambda_2=2\):
		\begin{equation}
		A-2I =
		\begin{pmatrix}2&2\\1&1\end{pmatrix}.
		\end{equation}
	Nullspace: \(v_1=-v_2\), so:
		\begin{equation}
		\vec{v}_2 =
		\begin{pmatrix}1\\-1\end{pmatrix}.
		\end{equation}
	\end{frame}
\end{document}