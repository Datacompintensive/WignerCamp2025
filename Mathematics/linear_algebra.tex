
\documentclass{beamer}
\usefonttheme{serif} % default family is serif
\usepackage{graphicx}
\usepackage{xcolor}
\usepackage{colortbl}
\usepackage{multirow,tabularx}
\usepackage{tikz}
\usepackage{hyperref}
\usepackage{physics}
\usepackage{siunitx}

%% For the name of the chemical compounds
%\usepackage{mhchem}

% Code listing
\usepackage{listings}
\usepackage{xcolor}

% Define Bash syntax highlighting
\lstdefinelanguage{python}{
  morekeywords={if, then, else, fi, for, while, do, done, exit, echo, function},
  morecomment=[l]{\#},
  morestring=[b]",
  sensitive=true
}

% Define listing style
\lstset{
  language=python,
  basicstyle=\ttfamily\small,
  keywordstyle=\color{blue}\bfseries,
  commentstyle=\color{gray},
  stringstyle=\color{red},
  numbers=left,
  numberstyle=\tiny\color{gray},
  stepnumber=1,
  numbersep=5pt,
  breaklines=true,
  breakatwhitespace=true,
  frame=single,
  backgroundcolor=\color{lightgray!20},
  tabsize=2
}

\usepackage{amsthm}
\usepackage{amsmath}
%\usepackage{multilined}

%\usefonttheme{professionalfonts} % using non standard fonts for beamer
%\usefonttheme{serif} % default family is serif
%\usepackage{fontspec}
%\setmainfont{Times New Roman}

% Redefine example environment with separate numbering
\newtheorem{examplex}{Example}
\renewenvironment{example}[1][]{%
  \setbeamercolor{block title example}{use=structure,fg=white,bg=structure.fg!75!black}%
  \setbeamercolor{block body example}{parent=normal text,use=block title,bg=block title.bg!10!bg}%
  \begin{examplex}[#1]%
}{%
  \end{examplex}%
}
% For legend of equations
\newenvironment{conditions*}
  {\par{\abovedisplayskip}\noindent
   \tabularx{\columnwidth}{>{$}l<{$} @{\ : } >{\raggedright\arraybackslash}X}}
  {\endtabularx\par\vspace{\belowdisplayskip}}

% Definition block formatting remains the same
\setbeamertemplate{theorems}[numbered]
\setbeamercolor{block title}{use=structure,fg=white,bg=structure.fg!75!black}
\setbeamercolor{block body}{parent=normal text,use=block title,bg=block title.bg!10!bg}

% Package for figure captions
\usepackage[font=small,skip=2pt]{caption}
% Decrease space/padding after figure captions
\setlength{\belowcaptionskip}{-10pt}

% Define blue color
\definecolor{purple}{rgb}{0.3, 0.0, 0.557}
\definecolor{blue}{rgb}{0.0234375, 0.11328125, 0.609375}

\usepackage{multicol}
\usepackage{wrapfig}

% Use blue theme for the presentation
\setbeamercolor{frametitle}{fg=blue}
\setbeamercolor{title}{fg=blue}
\setbeamercolor{structure}{fg=blue}

\newcommand{\circumeq}{\mathrel{\widehat{=}}}
\newcommand\scalemath[2]{\scalebox{#1}{\mbox{\ensuremath{\displaystyle #2}}}}

% Define affiliations
\newcommand{\affiliation}[2]{#1\textsuperscript{\textcolor{blue}{#2}}}

\title{\textcolor{blue}{Introduction to Linear Algebra}}
\subtitle{\textcolor{blue}{Wigner Summer Camp \\ Data and Compute Intensive Sciences Research Group}}
\author{\textcolor{blue}{Balázs, Paszkál, Vince, Levente, Antal \\ Éva, Hajni}}

\date{\textcolor{blue}{7-11 July 2025}}


% Redefine footline to show only the slide number on the bottom left in blue
\setbeamertemplate{footline}
{
  \leavevmode%
  \hbox{%
  \begin{beamercolorbox}[wd=.1\paperwidth,ht=2.25ex,dp=1ex,left]{page number in head/foot}%
    \hspace{1em} \textcolor{blue}{\insertframenumber}%
  \end{beamercolorbox}}%Wigner Summer Camp

  \vskip0pt%
}


\begin{document}

\begin{frame}
  \titlepage
  \begin{columns}
    \centering
    \column{0.3\textwidth}
    %\includegraphics[width=0.7\textwidth]{img/logo.png}
    \column{0.3\textwidth}
    \centering
    \includegraphics[width=0.8\textwidth]{img/logo.png}
    \column{0.3\textwidth}
    %\includegraphics[width=0.7\textwidth]{img/logo.png}
    \centering
  \end{columns}
\end{frame}

\begin{frame}{Introduction}
    Linear algebra deals with:
    \begin{itemize}
        \item Vectors.
        \item Matrices. (The linear transformations of vector spaces.)
    \end{itemize}

    It is used in many places:
    \begin{itemize}
        \item Physics (velocity, force, ...).
        \item Artificial intelligence - neural networks.
        \item \textbf{Adaptive Law-Based Transformation (ALT).} 
        \item ...
    \end{itemize}
\end{frame}

\begin{frame}{What is a Vector?}
  \begin{itemize}
    \item \textbf{Mathematician's view:} A set that satisfies the 8 axioms.
    \item \textbf{Physicist's view:} An arrow in space with direction and magnitude.
    \item \textbf{Computer Scientist's view:} A 1D array or list of numbers.
    \begin{equation}
    \vec{v} = \begin{bmatrix} 3 \\ -1 \\ 2 \end{bmatrix}, \quad
    \mathbf{w} = \langle 4, 0, -5 \rangle, \quad
    \vec{u} = (1, 2, 3).
    \end{equation}
    \item \textbf{Notation:}
    \begin{itemize}
      \item Boldface: $\mathbf{v}$ (common in CS and mathematical books)
      \item Arrow: $\vec{v}$ (common in physics)
      \item Angled brackets or parentheses
    \end{itemize}
  \end{itemize}
\end{frame}

\begin{frame}{Example vectors}
    \begin{example}[Vectors]
        \begin{equation}
        \vec{a} = \begin{bmatrix} 1 \\ 0 \end{bmatrix}, \quad
        \vec{b} = \begin{bmatrix} 1 \\ 1 \end{bmatrix}, \quad
        \vec{c} = \begin{bmatrix} -3 \\ 2 \\ 5 \end{bmatrix}.
        \end{equation}
    \end{example}
    \begin{example}[Common vectors]
        \begin{equation}
            \vec{i} = \begin{bmatrix} 1 \\ 0 \end{bmatrix}, \quad
            \vec{j} = \begin{bmatrix} 0 \\ 1 \end{bmatrix}, \quad
            \vec{0} = \begin{bmatrix} 0 \\ 0 \end{bmatrix}.
        \end{equation}
    \end{example}
\end{frame}

\begin{frame}{What is the Dot Product?}
  \begin{itemize}
    \item The dot product of two vectors $\vec{a}$ and $\vec{b}$ is:
    \begin{equation}
    \vec{a} \cdot \vec{b} = a_1 b_1 + a_2 b_2 + \dots + a_n b_n.
    \end{equation}
    \item It can be shown that:
    \begin{equation}
        \vec{a} \cdot \vec{b} = ||\vec{a}||\cdot||\vec{b}||\cos(\angle\left(\vec{a},\vec{b}\right)).
    \end{equation}
    \begin{example}[Dot product]    
        \begin{equation}
        \begin{bmatrix} 2 \\ 3 \end{bmatrix} \cdot \begin{bmatrix} 4 \\ -1 \end{bmatrix} = 2\cdot4 + 3\cdot(-1) = 8 - 3 = 5.
        \end{equation}
    \end{example}
  \end{itemize}
\end{frame}

\begin{frame}{Dot Product – Practice Exercises}
  \begin{itemize}
    \item Calculate the dot product of the following vector pairs:
    \begin{enumerate}
      \item $\begin{bmatrix} 1 \\ -2 \\ 3 \end{bmatrix} \cdot \begin{bmatrix} 4 \\ 0 \\ -1 \end{bmatrix}.$
      \item $\begin{bmatrix} 2 \\ 5 \end{bmatrix} \cdot \begin{bmatrix} -1 \\ 1 \end{bmatrix}.$
    \end{enumerate}
  \end{itemize}
\end{frame}

\begin{frame}{Dot Product – Solutions}
  \begin{itemize}
    \item \textbf{Solution to (a):}
    \[
    1 \cdot 4 + (-2) \cdot 0 + 3 \cdot (-1) = 4 + 0 - 3 = \boxed{1}.
    \]
    \item \textbf{Solution to (b):}
    \[
    2 \cdot (-1) + 5 \cdot 1 = -2 + 5 = \boxed{3}.
    \]
    \item Interpretation:
    \begin{itemize}
      \item (a) Zero \(\Rightarrow\) the vectors are perpendicular.
      \item (b) Positive \(\Rightarrow\) angle between the vectors is acute.
    \end{itemize}
  \end{itemize}
\end{frame}

\begin{frame}{What is a Matrix?}
  \begin{itemize}
    \item A 2D array of numbers arranged in rows and columns:
    \begin{equation}
    A = \begin{bmatrix} 1 & 2 & 3 \\ 4 & 5 & 6 \end{bmatrix},
    \end{equation}
    where \(A\) is a \(2\cross3\) matrix.
    \item Types:
    \begin{itemize}
      \item \textbf{Square:} same number of rows and columns (e.g., $2\times2$, $3\times3$). Mathematicians like them.
      \item \textbf{Non-square:} different number of rows and columns (e.g., $2\times3$, $3\times2$).
    \end{itemize}
  \end{itemize}
\end{frame}

\begin{frame}{Vector-Matrix Multiplication}
  \begin{itemize}
    \item Each row of the matrix is dotted with the vector:
    \begin{equation}
    A = \begin{bmatrix} 1 & 2 \\ 3 & 4 \\ 5 & 6 \end{bmatrix}, \quad \vec{v} = \begin{bmatrix} 2 \\ 1 \end{bmatrix}.
    \end{equation}
    \item Multiplication:
    \begin{equation}
    A\vec{v} = \begin{bmatrix}
    1\cdot2 + 2\cdot1 \\
    3\cdot2 + 4\cdot1 \\
    5\cdot2 + 6\cdot1
    \end{bmatrix} = \begin{bmatrix} 4 \\ 10 \\ 16 \end{bmatrix}.
    \end{equation}
  \end{itemize}
\end{frame}

\begin{frame}{Vector-Matrix Multiplication – Exercises}
  \begin{itemize}
    \item Multiply the vector $\vec{v}$ with matrix $M$ in the following examples:
    \begin{enumerate}
      \item General case:
      \begin{equation}
      \vec{v} = \begin{bmatrix} 1 \\ 2 \end{bmatrix},\quad
      M = \begin{bmatrix} 3 & 0 & 1 \\ 1 & 4 & 2 \end{bmatrix}.
      \end{equation}
      \item Interesting effect:
      \begin{equation}
      \vec{v} = \begin{bmatrix} 1 \\ 1 \end{bmatrix},\quad
      M_1 = \begin{bmatrix} 2 & 0 \\ 0 & 2 \end{bmatrix}, \quad
      M_2 = \begin{bmatrix} -2 & -2 \\ -2 & -2 \end{bmatrix}.
      \end{equation}
      \item Even more interesting effect:
      \begin{equation}
      \vec{v} = \begin{bmatrix} 1 \\ 1 \end{bmatrix},\quad
      M_3 = \begin{bmatrix} 1 & -1 \\ 2 & -2 \end{bmatrix}, \quad
      M_4 = \begin{bmatrix} 5 & -5 \\ -3 & 3 \end{bmatrix}.
      \end{equation}
    \end{enumerate}
    \item Try to interpret the geometric meaning of each result.
  \end{itemize}
\end{frame}

\begin{frame}{Vector-Matrix Multiplication – Solutions}
  \begin{enumerate}
    \item General case:
    \begin{equation}
    \vec{v} \cdot M = \begin{bmatrix} 1 \\ 2 \end{bmatrix} \cdot
    \begin{bmatrix} 3 & 0 & 1 \\ 1 & 4 & 2 \end{bmatrix}
    = \begin{bmatrix} 5 \\ 8 \\ 5 \end{bmatrix}.
    \end{equation}
    \item Stretching cases:
    \begin{equation}
    \vec{v} \cdot M_1 = \begin{bmatrix} 2 \\ 2 \end{bmatrix}=2\vec{v}, \quad
    \vec{v} \cdot M_2 = \begin{bmatrix} -4 \\ -4 \end{bmatrix} = -4\vec{v}.
    \end{equation}
    \item Nearly null output:
    \begin{equation}
    \vec{v} \cdot M_3 = \begin{bmatrix} 0 \\ 0 \end{bmatrix}=0\vec{v}, \quad
    \vec{v} \cdot M_4 = \begin{bmatrix} 0 \\ 0 \end{bmatrix}=0\vec{v}.
    \end{equation}
  \end{enumerate}
\end{frame}

\begin{frame}{Matrices as linear transformations}

\begin{itemize}
    \item Each matrix denotes a transformation $\vec{v} \mapsto M\vec{v}$. It's always linear.
    \item It can be proven that all linear transformations can be described with a matrix.
    
\end{itemize}
    
\end{frame}

\begin{frame}{Transformation examples}
  \begin{itemize}
    \item Stretching: 
    %\begin{equation}
    $\begin{bmatrix} 2 & 0 \\ 0 & 2 \end{bmatrix}.$ %\vec{v} = %\text{doubles length}
    %\end{equation}
    \item Rotation:
    %\begin{equation}
    %\text{Rotation by } 90^\circ: 
    $\begin{bmatrix} 0 & -1 \\ 1 & 0 \end{bmatrix}.$
    %\end{equation}
    \item Reflection:
    %\begin{equation}
    %\text{Reflection over x-axis: }
    $\begin{bmatrix} 1 & 0 \\ 0 & -1 \end{bmatrix}.$
    %\end{equation}
    \item Projection (flattening onto a line or plane) $\begin{bmatrix} 1 & 0 \\ 0 & 0 \end{bmatrix}.$
  \end{itemize}
\end{frame}

\begin{frame}{Matrix-Matrix Multiplication}
  Procedure is the same as the Vector-Matrix multiplication. If an \(n\cross m\) matrix is multiplied by an \(m\cross k\), the result is an \(n\cross k\) matrix.
  \begin{itemize}
  	\item The number of columns in the first matrix must be equal to the number of rows in the second matrix.
  \end{itemize}
  \begin{equation} A = \begin{bmatrix} a_{11} & a_{12} \\ a_{21} & a_{22} \end{bmatrix}, \quad B = \begin{bmatrix} b_{11} & b_{12} \\ b_{21} & b_{22} \end{bmatrix} \end{equation}

  \begin{equation} A\cdot B =  \begin{bmatrix} {a_{11} \cdot b_{11} + a_{12} \cdot b_{21}} & {a_{11} \cdot b_{12} + a_{12} \cdot b_{22}} \\ {a_{21} \cdot b_{11} + a_{22} \cdot b_{21}} & {a_{21} \cdot b_{12} + a_{22} \cdot b_{22}} \end{bmatrix} \end{equation}

  \begin{example}
	\begin{equation}
		A = \begin{bmatrix} 1 & 2 \\ 3 & 4 \end{bmatrix}, \quad B = \begin{bmatrix} 2 & 0 \\ 1 & 2 \end{bmatrix}
		\Rightarrow A\cdot B = \begin{bmatrix} 4 & 4 \\ 10 & 8 \end{bmatrix}
	\end{equation}
\end{example}
\end{frame}

\begin{frame}{Other examples}
  \begin{example}
    \begin{equation}
      A = \begin{bmatrix} 1 & 2 & 3 \\ 4 & 5 & 6 \end{bmatrix}, B = \begin{bmatrix} 1 & 0 \\ 0 & 1 \\ 1 & 1 \end{bmatrix}
      \Rightarrow A\cdot B = \begin{bmatrix} 4 & 5 \\ 10 & 11 \end{bmatrix}
    \end{equation}
  \end{example}
  \begin{example}
    \begin{equation}
      A = \begin{bmatrix} 8 & 4 & 11 & 5 \\ 4 & 4 & 21 & 6 \\ 6 & 9 & 11 & 8 \end{bmatrix}, B = \begin{bmatrix} 3 & 2 \\ 4 & 6 \\ 1 & 4 \\ 7 & 2 \end{bmatrix}
      \Rightarrow A\cdot B = \begin{bmatrix} 86 & 94 \\ 91 & 128 \\ 121 & 126\end{bmatrix}
    \end{equation}
  \end{example}
\end{frame}

\begin{frame}{Eigenvalues and Eigenvectors}
  \begin{itemize}
    \item Notice that 
    \begin{equation}
        \begin{bmatrix} 2 & 0 \\ 0 & 2 \end{bmatrix}\begin{bmatrix} 1 \\ 1 \end{bmatrix} = 2\begin{bmatrix} 1 \\ 1 \end{bmatrix}. 
    \end{equation}
    \item Generally for a square matrix \(A\) if
    \begin{equation}
        A\vec{v} = \lambda \vec{v},
    \end{equation}
    then:
    \begin{itemize}
      \item $\vec{v}$ is an eigenvector and
      \item scalar $\lambda$ is the corresponding eigenvalue.
    \end{itemize}
    \item These vectors don't change direction when multiplied by $A$.
  \end{itemize}
\end{frame}

\begin{frame}{Eigenvalues and Eigenvectors – Properties}
  \begin{itemize}
    \item If $A \vec{v} = \lambda \vec{v}$, then $\vec{v}$ is an eigenvector of $A$ with eigenvalue $\lambda$.
    \item Eigenvectors are defined up to a scalar: if $\vec{v}$ is an eigenvector, so is $c\vec{v}$ for any $c \neq 0$.
    \item The set of all eigenvectors corresponding to a single eigenvalue forms a vector subspace.
    \item A square matrix of size $n \times n$ has at most $n$ eigenvalues (including complex ones and multiplicities).
    \item If all eigenvalues of a matrix are positive, the matrix is positive definite.
    \item Diagonal matrices (matrices where for all $a_{ij}=0$ where $i\neq j$) have their diagonal elements as eigenvalues, and the standard basis vectors as eigenvectors.
    \item The eigenvalues of a complex Hermitian or real symmetric matrix are real.
  \end{itemize}
\end{frame}

\begin{frame}[fragile]{Finding eigenvalues}
Using a computer:
\begin{lstlisting}
import torch
A = torch.tensor([[2., 0.], [0., 2.]])
eigenvalues, eigenvectors = torch.linalg.eigh(A)
print("Eigenvalues:", eigenvalues)
print("Eigenvectors:\n", eigenvectors)
\end{lstlisting}
\end{frame}

\begin{frame}{Revision}
In this presentation you learned about:
    \begin{itemize}
        \item Vectors.
        \item Matrices.
        \item Dot products.
        \item Vector-Matrix and Matrix-Matrix multiplication.
        \item Eigendecomposition.
    \end{itemize}
    
\end{frame}

\begin{frame}{What is the Determinant?}
\begin{itemize}
  \item The determinant is a scalar value associated with a square matrix \( A \).
  \item It plays a key role in linear algebra:
  \begin{itemize}
    \item \( A \) is invertible if and only if \( \det(A) \neq 0 \).
    \item \( |\det(A)| \) equals the volume scaling factor of the linear transformation defined by \( A \).
  \end{itemize}
  \item Determinants are widely used in solving linear systems, computing eigenvalues, and understanding geometric transformations.
\end{itemize}
\end{frame}

\begin{frame}{Determinant of a \(2 \times 2\) Matrix}
\begin{itemize}
  \item For a \(2 \times 2\) matrix
  \begin{equation}
    A =
    \begin{pmatrix} a & b \\ c & d \end{pmatrix},
  \end{equation}
  the determinant is given by
  \begin{equation}
    \det(A) = ad - bc. \label{eq:det2x2}
  \end{equation}
  \item Example:
  \begin{equation}
    \det
    \begin{pmatrix} 1 & 2 \\ 3 & 4 \end{pmatrix} =
    (1)(4) - (2)(3) = -2. \label{eq:det2x2_example}
  \end{equation}
\end{itemize}
\end{frame}

\begin{frame}{Exercise: \(2 \times 2\) Determinant}
\begin{example}[Exercise]
Compute the determinant of the matrix:
\begin{equation}
  A =
  \begin{pmatrix} 5 & 7 \\ 2 & 6 \end{pmatrix}.
\end{equation}
\end{example}
\end{frame}

\begin{frame}{Solution: \(2 \times 2\) Determinant}
\begin{example}[Solution]
Using Equation~\eqref{eq:det2x2}:
\begin{equation}
  \det(A) =
  (5)(6) - (7)(2) =
  30 - 14 =
  16.
\end{equation}
\end{example}
\end{frame}

\begin{frame}{Determinant of a \(3 \times 3\) Matrix}
We compute the determinant of a \(3 \times 3\) matrix by expanding along the first row:
\begin{equation}
A =
\begin{pmatrix} 
a & b & c \\ 
d & e & f \\ 
g & h & i 
\end{pmatrix}.
\end{equation}

The formula is:
\begin{equation}
\det(A) =
a 
\begin{vmatrix} 
e & f \\ h & i 
\end{vmatrix}
- b 
\begin{vmatrix} 
d & f \\ g & i 
\end{vmatrix}
+ c 
\begin{vmatrix} 
d & e \\ g & h 
\end{vmatrix}. \label{eq:det3x3_expansion}
\end{equation}

Each \(2 \times 2\) minor is the determinant of the smaller matrix obtained by deleting the corresponding row and column.
\end{frame}

\begin{frame}{Why does this work?}
The determinant of \(A\) is defined recursively by expanding into \(2 \times 2\) minors:
\begin{equation}
\begin{vmatrix} 
m_{11} & m_{12} \\ m_{21} & m_{22} 
\end{vmatrix} =
m_{11}m_{22} - m_{12}m_{21}.
\end{equation}

So:
\begin{align}
\det(A) =
a 
\begin{vmatrix} 
e & f \\ h & i 
\end{vmatrix}
&- b 
\begin{vmatrix} 
d & f \\ g & i 
\end{vmatrix}
+ c 
\begin{vmatrix} 
d & e \\ g & h 
\end{vmatrix}. 
\end{align}

\vspace{0.2cm}
This is called the \emph{Laplace expansion}.

Signs come from the cofactor sign pattern:
\begin{center}
\begin{tabular}{|c|c|c|}
\hline
\cellcolor{blue!20} \(+\) & \cellcolor{red!20} \(-\) & \cellcolor{blue!20} \(+\) \\ \hline
\cellcolor{red!20} \(-\) & \cellcolor{blue!20} \(+\) & \cellcolor{red!20} \(-\) \\ \hline
\cellcolor{blue!20} \(+\) & \cellcolor{red!20} \(-\) & \cellcolor{blue!20} \(+\) \\ \hline
\end{tabular}
\end{center}
The determinant of an \(n \times n\) matrix is computed similarly: expanding along any row or column into \((n-1) \times (n-1)\) minors.
\end{frame}

\begin{frame}{Numerical Example}
Compute:
\begin{equation}
A =
\begin{pmatrix} 
1 & 2 & 3 \\ 
4 & 5 & 6 \\ 
7 & 8 & 9 
\end{pmatrix}.
\end{equation}

Expanding along the first row:
\begin{align}
\det(A) &=
1 
\begin{vmatrix} 
5 & 6 \\ 8 & 9 
\end{vmatrix}
- 2 
\begin{vmatrix} 
4 & 6 \\ 7 & 9 
\end{vmatrix}
+ 3 
\begin{vmatrix} 
4 & 5 \\ 7 & 8 
\end{vmatrix} \\
&= 1(5\cdot9 -6\cdot8) -2(4\cdot9 -6\cdot7) +3(4\cdot8 -5\cdot7) \\
&= 1(-3) -2(-6) +3(-3) \\
&= -3 +12 -9 = 0.
\end{align}

The determinant of this matrix is \(0\), so it is singular (non-invertible).
\end{frame}

\begin{frame}{Exercise: Compute a \(3 \times 3\) Determinant}
\begin{block}{Exercise}
Compute the determinant of:
\begin{equation}
A =
\begin{pmatrix} 
2 & 1 & 3 \\ 
0 & -1 & 4 \\ 
5 & 2 & 0 
\end{pmatrix}.
\end{equation}
Use expansion along the first row.
\end{block}
\end{frame}


\begin{frame}{Solution: Exercise}
\begin{block}{Solution}
\begin{align}
\det(A) &=
2 
\begin{vmatrix} 
-1 & 4 \\ 2 & 0 
\end{vmatrix}
- 1 
\begin{vmatrix} 
0 & 4 \\ 5 & 0 
\end{vmatrix}
+ 3 
\begin{vmatrix} 
0 & -1 \\ 5 & 2 
\end{vmatrix} \\
&= 2((-1)(0) -4(2)) -1(0 -4(5)) +3((0)(2) -(-1)(5)) \\
&= 2(0-8) -1(0-20) +3(0+5) \\
&= -16 +20 +15 = 19.
\end{align}
Therefore, \(\det(A) = 19.\)
\end{block}
\end{frame}

\begin{frame}{The Eigenvalue Problem — Step by Step}
We want to find scalars \(\lambda\) and nonzero vectors \(\vec{v}\) such that:
\begin{equation}
A \vec{v} = \lambda \vec{v}.
\end{equation}

\textbf{Step 1:} Rewrite as
\begin{equation}
(A - \lambda I)\vec{v} = 0.
\end{equation}

\textbf{Step 2:} For a nontrivial solution (\(\vec{v} \neq 0\)), the matrix \(A - \lambda I\) must be singular:
\begin{equation}
\det(A - \lambda I) = 0.
\end{equation}

This is called the \emph{characteristic equation}.
\end{frame}

\begin{frame}{Solving the Eigenvalue Problem}
\textbf{Step 3:} Expand \(\det(A - \lambda I) = 0\) to get a polynomial in \(\lambda\).

\textbf{Step 4:} The roots of this polynomial are the eigenvalues.

\textbf{Step 5:} For each eigenvalue \(\lambda\), solve:
\begin{equation}
(A - \lambda I)\vec{v} = 0
\end{equation}
to find the corresponding eigenvector(s) by computing the nullspace.
\end{frame}

\begin{frame}{Example: Eigenvalues and Eigenvectors}
Given:
\begin{equation}
A =
\begin{pmatrix} 2 & 1 \\ 1 & 2 \end{pmatrix}.
\end{equation}

\textbf{Step 1:} Compute \(A - \lambda I\):
\begin{equation}
A - \lambda I =
\begin{pmatrix} 2-\lambda & 1 \\ 1 & 2-\lambda \end{pmatrix}.
\end{equation}

\textbf{Step 2:} Characteristic equation:
\begin{align}
\det(A-\lambda I) &= (2-\lambda)(2-\lambda) -1 \\
&= \lambda^2 -4\lambda+3 =0.
\end{align}

\textbf{Step 3:} Roots:
\begin{equation}
\lambda_1 = 1, \quad \lambda_2 = 3.
\end{equation}
\end{frame}

\begin{frame}{Example: Eigenvectors}
For \(\lambda_2=3\):
\begin{equation}
A-3I =
\begin{pmatrix} -1 & 1 \\ 1 & -1 \end{pmatrix}.
\end{equation}
We solve:
\begin{equation}
(A-3I)\vec{v} =
\begin{pmatrix} -1 & 1 \\ 1 & -1 \end{pmatrix}
\vec{v} =
0.
\end{equation}

This gives \(v_1=v_2\), so an eigenvector is:
\begin{equation}
\vec{v}_1 =
\begin{pmatrix}1\\1\end{pmatrix}.
\end{equation}

For \(\lambda_1=1\):
\begin{equation}
A-I =
\begin{pmatrix}1 &1 \\1 &1\end{pmatrix}.
\end{equation}

This gives \(v_1=-v_2\), so an eigenvector is:
\begin{equation}
\vec{v}_2 =
\begin{pmatrix}1\\-1\end{pmatrix}.
\end{equation}
\end{frame}

\begin{frame}{Exercise: Find Eigenvalues and Eigenvectors}
\begin{block}{Exercise}
Find the eigenvalues and one eigenvector for each eigenvalue of the matrix:
\begin{equation}
A =
\begin{pmatrix} 4 & 2 \\ 1 & 3 \end{pmatrix}.
\end{equation}
\end{block}
\end{frame}

\begin{frame}{Solution: Exercise}
\textbf{Step 1:} Compute characteristic polynomial:
\begin{align}
\det(A-\lambda I) &=
\begin{vmatrix}4-\lambda &2 \\1 &3-\lambda\end{vmatrix} =
(4-\lambda)(3-\lambda)-2 \\
&= \lambda^2 -7\lambda+10=0.
\end{align}

\textbf{Step 2:} Roots:
\begin{equation}
\lambda_1=5, \quad \lambda_2=2.
\end{equation}

\textbf{Step 3:} For \(\lambda_1=5\):
\begin{equation}
A-5I =
\begin{pmatrix}-1&2\\1&-2\end{pmatrix}.
\end{equation}
Nullspace: \(v_1=2v_2\), so:
\begin{equation}
\vec{v}_1 =
\begin{pmatrix}2\\1\end{pmatrix}.
\end{equation}

\end{frame}
\begin{frame}{Solution 2}
For \(\lambda_2=2\):
\begin{equation}
A-2I =
\begin{pmatrix}2&2\\1&1\end{pmatrix}.
\end{equation}
Nullspace: \(v_1=-v_2\), so:
\begin{equation}
\vec{v}_2 =
\begin{pmatrix}1\\-1\end{pmatrix}.
\end{equation}
\end{frame}



\end{document}
