\documentclass{beamer}
\usefonttheme{serif}
\usepackage{graphicx}
\usepackage{xcolor}
\usepackage{colortbl}
\usepackage{multirow,tabularx}
\usepackage{tikz}
\usepackage{hyperref}
\usepackage{physics}
\usepackage{siunitx}

% Code listing
\usepackage{listings}
\lstdefinelanguage{python}{
  morekeywords={if, then, else, fi, for, while, do, done, exit, echo, function},
  morecomment=[l]{\#},
  morestring=[b]",
  sensitive=true
}
\lstset{
  language=python,
  basicstyle=\ttfamily\small,
  keywordstyle=\color{blue}\bfseries,
  commentstyle=\color{gray},
  stringstyle=\color{red},
  numbers=left,
  numberstyle=\tiny\color{gray},
  stepnumber=1,
  numbersep=5pt,
  breaklines=true,
  breakatwhitespace=true,
  frame=single,
  backgroundcolor=\color{lightgray!20},
  tabsize=2
}

\usepackage{amsthm}
\usepackage{amsmath}

% Define blue color
\definecolor{purple}{rgb}{0.3, 0.0, 0.557}
\definecolor{blue}{rgb}{0.0234375, 0.11328125, 0.609375}

% Theme setup
\setbeamercolor{frametitle}{fg=blue}
\setbeamercolor{title}{fg=blue}
\setbeamercolor{structure}{fg=blue}

\title{\textcolor{blue}{Programming Tools for AI \& Data Science}}
\subtitle{\textcolor{blue}{Wigner Summer Camp \\ Data and Compute Intensive Sciences Research Group}}
\author{\textcolor{blue}{Balázs, Paszkál, Vince, Levente, Antal \\ Éva, Hajni}}
\date{\textcolor{blue}{7-11 July 2025}}

% Footline
\setbeamertemplate{footline}{
  \leavevmode%
  \hbox{%
  \begin{beamercolorbox}[wd=.1\paperwidth,ht=2.25ex,dp=1ex,left]{page number in head/foot}%
    \hspace{1em} \textcolor{blue}{\insertframenumber}%
  \end{beamercolorbox}}%
  \vskip0pt%
}

\begin{document}

\begin{frame}
  \titlepage
  \begin{columns}
    \column{0.3\textwidth}
    \column{0.3\textwidth}
    \centering
    \includegraphics[width=0.8\textwidth]{img/logo.png}
    \column{0.3\textwidth}
  \end{columns}
\end{frame}

%-----------------------------
\section{Introduction}
%-----------------------------

\begin{frame}{Why These Tools?}
  \begin{itemize}
    \item \textbf{Python}: Foundation for all other tools (syntax, control flow).
    \item \textbf{NumPy}: Efficient numerical operations (linear algebra, arrays).
    \item \textbf{PyTorch}: Deep learning framework (GPU acceleration, autograd).
    \item \textbf{Pandas}: Data manipulation (cleaning, analysis).
    \item \textbf{Matplotlib}: Visualization (plots, graphs).
  \end{itemize}
  \vspace{1em}
  \centering
  \includegraphics[width=0.7\textwidth]{img/ai_stack.png}  % Replace with your own diagram
\end{frame}

%-----------------------------
\section{Python Basics}
%-----------------------------

\begin{frame}{Python Basics}
  \begin{itemize}
    \item \textbf{What it covers}:
      \begin{itemize}
        \item Variables, loops, functions.
        \item Lists, dictionaries, comprehensions.
      \end{itemize}
    \item \textbf{AI relevance}:
      \begin{itemize}
        \item Core syntax for implementing algorithms.
        \item Data structures for preprocessing.
      \end{itemize}
  \end{itemize}
  \begin{example}[List Comprehension]
    \begin{lstlisting}
    squares = [x**2 for x in range(10)]  # [0, 1, 4, ..., 81]
    \end{lstlisting}
  \end{example}
\end{frame}

%-----------------------------
\section{NumPy}
%-----------------------------

\begin{frame}{NumPy: Numerical Computing}
  \begin{itemize}
    \item \textbf{What it does}:
      \begin{itemize}
        \item Multi-dimensional arrays (\texttt{ndarray}).
        \item Broadcasting, vectorized operations.
      \end{itemize}
    \item \textbf{AI relevance}:
      \begin{itemize}
        \item Basis for PyTorch/TensorFlow.
        \item Fast matrix operations (e.g., dot products).
      \end{itemize}
  \end{itemize}
  \begin{example}[Matrix Multiplication]
    \begin{lstlisting}
    import numpy as np
    A = np.array([[1, 2], [3, 4]])
    B = np.array([[5, 6], [7, 8]])
    C = A @ B  # [[19, 22], [43, 50]]
    \end{lstlisting}
  \end{example}
\end{frame}

%-----------------------------
\section{PyTorch}
%-----------------------------

\begin{frame}{PyTorch: Deep Learning}
  \begin{itemize}
    \item \textbf{What it does}:
      \begin{itemize}
        \item Tensors with GPU support.
        \item Autograd for backpropagation.
      \end{itemize}
    \item \textbf{AI relevance}:
      \begin{itemize}
        \item Neural networks (CNNs, RNNs).
        \item Eigenvalue decomposition (e.g., PCA).
      \end{itemize}
  \end{itemize}
  \begin{example}[GPU Tensor]
    \begin{lstlisting}
    import torch
    x = torch.tensor([1., 2.], device='cuda')  # Move to GPU
    \end{lstlisting}
  \end{example}
\end{frame}

%-----------------------------
\section{Pandas}
%-----------------------------

\begin{frame}{Pandas: Data Wrangling}
  \begin{itemize}
    \item \textbf{What it does}:
      \begin{itemize}
        \item \texttt{DataFrame} (tabular data).
        \item Merging, filtering, grouping.
      \end{itemize}
    \item \textbf{AI relevance}:
      \begin{itemize}
        \item Preprocessing datasets (CSV, JSON).
        \item Feature engineering.
      \end{itemize}
  \end{itemize}
  \begin{example}[CSV Loading]
    \begin{lstlisting}
    import pandas as pd
    data = pd.read_csv('dataset.csv', sep=';')
    \end{lstlisting}
  \end{example}
\end{frame}

%-----------------------------
\section{Matplotlib}
%-----------------------------

\begin{frame}{Matplotlib: Visualization}
  \begin{itemize}
    \item \textbf{What it does}:
      \begin{itemize}
        \item Line plots, histograms, scatter plots.
        \item Customizable styles (colors, markers).
      \end{itemize}
    \item \textbf{AI relevance}:
      \begin{itemize}
        \item Debugging model performance.
        \item Visualizing embeddings (t-SNE, PCA).
      \end{itemize}
  \end{itemize}
  \begin{example}[Simple Plot]
    \begin{lstlisting}
    import matplotlib.pyplot as plt
    plt.plot([1, 2, 3], [4, 5, 6], 'ro-')
    plt.show()
    \end{lstlisting}
  \end{example}
\end{frame}

%-----------------------------
\section{Summary}
%-----------------------------

\begin{frame}{Key Takeaways}
  \begin{itemize}
    \item \textbf{Python}: Glue language for AI pipelines.
    \item \textbf{NumPy/PyTorch}: Math backbone (CPU/GPU).
    \item \textbf{Pandas}: Clean and structure data.
    \item \textbf{Matplotlib}: Communicate results.
  \end{itemize}
  \vspace{1em}
  \centering
  \includegraphics[width=0.6\textwidth]{img/workflow.png}  % Replace with your workflow diagram
\end{frame}

\end{document}