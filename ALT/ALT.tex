\documentclass{beamer}
\usefonttheme{serif} % default family is serif
\usepackage{graphicx}
\usepackage{xcolor}
\usepackage{colortbl}
\usepackage{multirow,tabularx}
\usepackage{tikz}
\usepackage{hyperref}
\usepackage{physics}
\usepackage{siunitx}
\usepackage{listings}
\usepackage{amsthm}
\usepackage{amsmath}
\usepackage[font=small,skip=2pt]{caption}
\setlength{\belowcaptionskip}{-10pt}
\definecolor{blue}{rgb}{0.0234375, 0.11328125, 0.609375}
\usepackage{multicol}
\usepackage{wrapfig}
\setbeamercolor{frametitle}{fg=blue}
\setbeamercolor{title}{fg=blue}
\setbeamercolor{structure}{fg=blue}

\title{\textcolor{blue}{Adaptive Law-Based Transformation (ALT)}}
\subtitle{\textcolor{blue}{Wigner Summer Camp \\ Data and Compute Intensive Sciences Research Group}}
\author{\textcolor{blue}{Bal\'azs, Paszk\'al, Vince, Levente, \\ Antal \\ \'Eva, Hajni}}
\date{\textcolor{blue}{7-11 July 2025}}

\setbeamertemplate{footline}{
  \leavevmode%
  \hbox{%
  \begin{beamercolorbox}[wd=.1\paperwidth,ht=2.25ex,dp=1ex,left]{page number in head/foot}%
    \hspace{1em} \textcolor{blue}{\insertframenumber}%
  \end{beamercolorbox}}%
  \vskip0pt%
}

\AtBeginSection[]{
  \begin{frame}
  \vfill
  \centering
  \begin{beamercolorbox}[sep=8pt,center,shadow=true,rounded=true]{title}
    \usebeamerfont{title}\insertsectionhead\par%
  \end{beamercolorbox}
  \vfill
  \end{frame}
}

\begin{document}

\begin{frame}
  \titlepage
  \begin{columns}
    \column{0.3\textwidth}
    \column{0.3\textwidth}
    \centering
    \includegraphics[width=0.8\textwidth]{img/logo.png}
    \column{0.3\textwidth}
  \end{columns}
\end{frame}

\begin{frame}{Table of Contents}
  \tableofcontents
\end{frame}

\section{Dataset}

\begin{frame}{Dataset}
  Consider the following dataset:
  \begin{itemize}
    \item Train instance from class \textbf{a}: \( a = [1, 1, 2, 3, 5] \).
    \item Train instance from class \textbf{b}: \( b = [2, 1, 5, 7, 17] \).
    \item Test instance from an unknown class: \( x = [2, 3, 5, 8, 13] \).
  \end{itemize}
  
  We now apply ALT to classify the test instance, i.e., determine to which class \( x \) belongs.
\end{frame}

\section{The detailed calculation}

\begin{frame}{Step 1: Extracting the Laws}
  We define parameters \( r = 3, \; l = 2, \; k = 1 \). {\color{gray}(Their meaning will be discussed later.)}

  From \( a = [1, 1, 2, 3, 5] \) we extract the following segments of length \( r = 3 \):
  \begin{equation}
    [1, 1, 2], \; [1, 2, 3], \; [2, 3, 5].
  \end{equation}

  Then we embed them to \( l \times l = 2 \times 2 \) matrices:
  \begin{equation}
    \begin{bmatrix} 1 & 1 \\ 1 & 2 \end{bmatrix}, \quad
    \begin{bmatrix} 1 & 2 \\ 2 & 3 \end{bmatrix}, \quad
    \begin{bmatrix} 2 & 3 \\ 3 & 5 \end{bmatrix}.
  \end{equation}

  From those, the laws are the eigenvectors corresponding to the smallest absolute eigenvalues:
  \begin{equation}
    \begin{bmatrix} -0.8507 \\ 0.5257 \end{bmatrix}, \quad
    \begin{bmatrix} -0.8507 \\ 0.5257 \end{bmatrix}, \quad
    \begin{bmatrix} -0.8507 \\ 0.5257 \end{bmatrix}.
  \end{equation}
\end{frame}

\begin{frame}{Step 1: Extracting the Laws (continued)}
  We combine the results into one matrix and repeat the process for all training instances (separately for each class).

  From class \textbf{a}, we obtain laws:
  \begin{equation}
    L_a = \begin{bmatrix} -0.8507 & -0.8507 & -0.8507 \\ 0.5257 & 0.5257 & 0.5257 \end{bmatrix}.
  \end{equation}

  From class \textbf{b}, we obtain laws:
  \begin{equation}
    L_b = \begin{bmatrix} -0.9571 & -0.8702 & -0.9085 \\ 0.2898 & 0.4927 & 0.4179 \end{bmatrix}.
  \end{equation}
\end{frame}

\begin{frame}{Step 2: Embedding the Test Instance}
  The test instance \( x = [2, 3, 5, 8, 13] \) is embedded as:
  \begin{equation}
    E = \begin{bmatrix} 2 & 3 \\ 3 & 5 \\ 5 & 8 \\ 8 & 13 \end{bmatrix}.
  \end{equation}
\end{frame}

\begin{frame}{Step 3: Projection by Laws}
  Multiplying with class \textbf{a} laws:
  \begin{equation}
    F_a = E L_a = \begin{bmatrix} -0.1241 & -0.1241 & -0.1241 \\ 0.0767 & 0.0767 & 0.0767 \\ -0.0474 & -0.0474 & -0.0474 \\ 0.0293 & 0.0293 & 0.0293 \end{bmatrix}.
  \end{equation}

  Multiplying with class \textbf{b} laws:
  \begin{equation}
    F_b = E L_b = \begin{bmatrix} -1.0448 & -0.2623 & -0.5635 \\ -1.4224 & -0.1471 & -0.6363 \\ -2.4672 & -0.4094 & -1.1997 \\ -3.8895 & -0.5565 & -1.8360 \end{bmatrix}.
  \end{equation}
\end{frame}

\begin{frame}{Step 4: Feature Calculation}
  Using the mean of all squared values:
  \begin{itemize}
    \item Class \textbf{a}: \( \text{mean} = 0.0061 \).
    \item Class \textbf{b}: \( \text{mean} = 2.5359 \).
  \end{itemize}

  \textbf{Conclusion:} The test instance is closer to class \textbf{a}.
\end{frame}

\section{The Parameters \( r, l, k \)}

\begin{frame}{Meaning of Parameters \( r, l, k \)}
  \begin{itemize}
    \item \( r \): Length of the original sequence window used for feature extraction.
    \item \( l \): Embedding size, defines a symmetric \( l \times l \) matrix.
    \item \( 2l - 1 \): Number of equally spaced elements sampled from the window of length \( r \).
    \item \( k \): Step size (stride) used for shifting the window over the time series.
  \end{itemize}
\end{frame}

\begin{frame}{Example \(r, l, k\) values}
    Consider the following instance: \(x' = [2, 1, 5, 7, 17, 31, 65, 127, 257, 511]\).
    
    Let \(r=5, l= 2, \) and \(k=1\). Then \(2l-1=3\) and
    \vspace{10px}
    
    \(x'_1 = [\boxed{\boxed{2}, 1, \boxed{5}, 7, \boxed{17}}, 31, 65, 127, 257, 511].\)
    \(x'_2 = [2, \boxed{\boxed{1}, 5, \boxed{7}, 17, \boxed{31}}, 65, 127, 257, 511].\)
    \(x'_2 = [2, 1, \boxed{\boxed{5}, 7, \boxed{17}, 31, \boxed{65}}, 127, 257, 511].\)
    \(x'_3 = [2, 1, 5, \boxed{\boxed{7}, 17, \boxed{31}, 65, \boxed{127}}, 257, 511].\)
    \(x'_4 = [2, 1, 5, 7, \boxed{\boxed{17}, 31, \boxed{65}, 127, \boxed{257}}, 511].\)
    \(x'_5 = [2, 1, 5, 7, 17, \boxed{\boxed{31}, 65, \boxed{127}, 257, \boxed{511}}].\)
    
\end{frame}

\begin{frame}{Example \(r, l, k\) values}
    Consider the following instance: \(x'' = [2, 1, 5, 7, 17, 31, 65, 127, 257, 511]\).
    
    Let \(r=7, l= 2, \) and \(k=2\). Then \(2l-1=3\) and
    \vspace{10px}
    
    \(x''_1 = [\boxed{\boxed{2}, 1, 5, \boxed{7}, 17, 31, \boxed{65}}, 127, 257, 511].\)
    \(x''_2 = [2, 1, \boxed{\boxed{5}, 7, 17, \boxed{31}, 65, 127, \boxed{257}}, 511].\)

    
\end{frame}

\begin{frame}{Summary of ALT Method}
  \begin{itemize}
    \item Extract subsequences and embed into symmetric matrices.
    \item Compute eigenvectors of smallest absolute eigenvalues — these are the laws.
    \item Multiply embedded test instances with class-specific laws.
    \item Compute summary statistics (e.g., mean of squared values).
    \item Classify by choosing the class that minimizes the statistic.
  \end{itemize}
\end{frame}

\end{document}